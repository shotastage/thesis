\appendix
\chapter{APPENDIX}


\section{本研究の実験に用いたDocker環境にインストールしたPythonパッケージ}

下記はpip freeze\footnote{pipはPythonのライブラリなどをインストールするパッケージマネージャー}コマンドの実行結果である. 本研究の実験に使用したDockerコンテナ上のOSには以下のPythonパッケージを用いている.

\begin{lstlisting}[caption = pip freezeコマンドの実行結果, label = program1]
absl-py==0.9.0
asn1crypto==0.24.0
astor==0.8.1
attrs==19.3.0
backcall==0.1.0
bleach==3.1.0
cachetools==4.0.0
certifi==2019.11.28
chardet==3.0.4
cloudpickle==1.2.2
cryptography==2.1.4
cycler==0.10.0
decorator==4.4.1
defusedxml==0.6.0
entrypoints==0.3
enum34==1.1.6
future==0.18.2
gast==0.2.2
google-auth==1.10.0
google-auth-oauthlib==0.4.1
google-pasta==0.1.8
grpcio==1.26.0
gym==0.15.4
h5py==2.10.0
idna==2.6
imageio==2.6.1
importlib-metadata==1.4.0
ipykernel==5.1.3
ipython==7.11.1
ipython-genutils==0.2.0
ipywidgets==7.5.1
jedi==0.15.2
Jinja2==2.10.3
jsonschema==3.2.0
jupyter==1.0.0
jupyter-client==5.3.4
jupyter-console==6.0.0
jupyter-core==4.6.1
jupyter-http-over-ws==0.0.7
Keras==2.3.1
Keras-Applications==1.0.8
Keras-Preprocessing==1.1.0
keras-rl2==1.0.3
keyring==10.6.0
keyrings.alt==3.0
kiwisolver==1.1.0
Markdown==3.1.1
MarkupSafe==1.1.1
matplotlib==3.1.2
mistune==0.8.4
more-itertools==8.0.2
nbconvert==5.6.1
nbformat==5.0.3
networkx==2.4
notebook==6.0.2
numpy==1.18.1
oauthlib==3.1.0
opencv-python==4.1.2.30
opt-einsum==3.1.0
pandocfilters==1.4.2
parso==0.5.2
pexpect==4.7.0
pickleshare==0.7.5
Pillow==7.0.0
prometheus-client==0.7.1
prompt-toolkit==2.0.10
protobuf==3.11.2
ptyprocess==0.6.0
pyasn1==0.4.8
pyasn1-modules==0.2.8
pycrypto==2.6.1
pyglet==1.3.2
Pygments==2.5.2
pygobject==3.26.1
pyparsing==2.4.6
pyrsistent==0.15.7
    python-dateutil==2.8.1
    PyWavelets==1.1.1
    pyxdg==0.25
    PyYAML==5.3
    pyzmq==18.1.1
    qtconsole==4.6.0
    requests==2.22.0
    requests-oauthlib==1.3.0
    rsa==4.0
    scikit-image==0.16.2
    scipy==1.4.1
    SecretStorage==2.3.1
    Send2Trash==1.5.0
    six==1.13.0
    tb-nightly==1.14.0a20190603
    tensorboard==2.1.0
    tensorflow==2.0.0b1
    tensorflow-estimator==2.1.0
    termcolor==1.1.0
    terminado==0.8.3
    testpath==0.4.4
    tf-estimator-nightly==1.14.0.dev2019060501
    tornado==6.0.3
    traitlets==4.3.3
    urllib3==1.25.7
    wcwidth==0.1.8
    webencodings==0.5.1
    Werkzeug==0.16.0
    widgetsnbextension==3.5.1
    wrapt==1.11.2
    zipp==0.6.0
\end{lstlisting}
    
    

\section{実験を行ったDockerコンテナの定義ファイル}

本研究では以下の定義ファイルを使用してDocker実験用環境\footnote{docker-compose upコマンドによりDockerfile, docker-compose.yml両定義ファイルに従ったコンテナが作成される}を構築した.

\begin{lstlisting}[caption = Dockerfileの定義, label = program1]
FROM tensorflow/tensorflow:latest-py3-jupyter
LABEL maintainer="shotastage"
RUN apt-get update -y
RUN apt-get install libsm6 libxrender1 libxext6 python-opengl -y
RUN pip install --upgrade pip
RUN pip install -q keras scikit-image keras-rl2 gym
RUN mkdir /catp/
RUN chmod 777 /catp/
\end{lstlisting}



\begin{lstlisting}[caption = docker-compose.ymlの定義, label = program1]
version: '3.7'
services:
    tensorflow:
        build:
            context: .
            dockerfile: Dockerfile
        container_name: tensorflow
        volumes:
            - ./dqncatp:/catp/dqncatp/
            - ./catpvalidator:/catp/catpvalidator/
            - ./catptools:/catp/catptools/
            - ./catpframework:/catp/catpframework/
        tty: true
        command: /bin/bash
        ports:
            - 8889:8889
    
\end{lstlisting}
    

