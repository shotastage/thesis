\chapter{実験}
\label{implementation}

本章では提案手法の実験について述べる.

\section{概要}

本研究では, 以下に述べるコンピュータープログラムを作成し, モビリティや利用者の満足度の変化をコンピューター上で再現する.
作成するプログラムは深層強化学習を行う学習器と環境定義プログラムである.

\section{構成}

この実験での各プログラムやミドルウェアの関係性を~\ref{system_overview}に示す.
pgAdmin~\cite{pgadmin}では, 地図データから選んだ道路の線形データを作成している. その後, 線形データはMap Vectorizerにて正方行列データに変換される.
正方行列に変換された道路データはCNN~\cite{CNN}に入力され, 行動がアウトプットとして出力される. アウトプットされた行動は環境構成プログラムεに入力される.
εはユースケースを定義したクラスと距離表現クラスに定義された内容を基に演算を行い報酬を決定する.

\begin{figure}[H]
  \centering  % 図を真ん中に配置
  \includegraphics[clip,width = 13.0cm]{assets/pgs.eps}
  \caption{システムの関係}  \label{system_overview}
\end{figure}




\begin{table}[h]
  \caption{実験に使用したソフトウェア一覧と役割}
  \label{table:SpeedOfLight}
  \centering
  \begin{tabular}{clll}
    \hline
      名称 & 役割 \\
      \hline \hline
      PostgreSQL~\cite{pgsql} & 道路データの管理 \\
      PostGIS~\cite{postgis} & 地理データの表現と格納 \\
      pgAdmin~\cite{pgadmin} & 地理データのビジュアライズ \\
      QGIS~\cite{qgis} & Open Street Mapの情報読み込み及地理データの閲覧 \\
    \hline
  \end{tabular}
\end{table}


\subsection{仮装道路モデルの作成}

本研究では, 実際の道路を参考にいくつかの道路を抜き出した仮装の道路網を作成した.
選択した道路の線形はPostgreSQL~\cite{pgsql}にてデータ化を行った. なお, この実験で使用したPostgreSQL~\cite{pgsql}には地理情報を扱う拡張であるPostGIS~\cite{postgis}をインストールしている.
データ化した内容は, 道路網の一定間隔の座標である. この座標同士を結ぶ線分をPOLYLINE型で記憶し路線名や道路が受け入れることのできる車の台数をCapacityとして定義した.


\begin{figure}[H]
  \centering  % 図を真ん中に配置
  \includegraphics[clip,width = 13.0cm]{assets/scn1.eps}
  \caption{pgadminにて道路データを地上に可視化した画面}  \label{system_overview}
\end{figure}



\begin{lstlisting}[caption = 路線データを表すクエリーの例, label = program1]

/*
    テーブル構造の定義
*/
CREATE TABLE geodb_catp_yokohama (
    id                  serial PRIMARY KEY,
    route_name          VARCHAR (100),
    has_spot    VARCHAR (20),
    capacity            INTEGER
);

/*
    線地理情報を記録するPOLYLINE型のカラムを追加
*/
SELECT AddGeometryColumn ('public', 'geodb_catp_yokohama', 'geometry_data', 4326, 'LINESTRING', 2);

/*
    道路の線形を表すデータのレコード
*/
INSERT INTO geodb_catp_yokohama (route_name, has_spot, capacity, geometry_data)
VALUES ('K1',
        'none',
        200,
        ST_GeomFromText('LINESTRING(139.633639 35.445911, 139.632631 35.447126 ...... 139.632323 35.472085)', 4326));
\end{lstlisting}



\section{Deep Q Neural Network Model}

この実験ではDeep Q Nueral Networkによる強化学習を行う学習器を構築するプログラムをPython言語を用いて作成した.
このプログラムは深層強化学習を形成するニューラルネットワークを定義したプログラムである.
ライブラリとしてTensorFlow及びKerasを用いた.


\begin{lstlisting}[caption = DQNの深層学習部分を形成するモデル, label = program1]
  from tensorflow.keras.models import Sequential
  from tensorflow.keras.layers import (
    Dense,
    Activation,
    Flatten,
    Convolution2D,
    Permute
  )
  from tensorflow.keras.optimizers import Adam
  import tensorflow.keras.backend as K
  
  
  def dqnmodel():
      model = Sequential()
  
      model.add(Permute((17, 17, 1), input_shape=input_shape))
  
      model.add(Convolution2D(32, (8, 8), strides=(4, 4)))
      model.add(Activation('relu'))
      model.add(Convolution2D(64, (4, 4), strides=(2, 2)))
      model.add(Activation('relu'))
      model.add(Convolution2D(64, (3, 3), strides=(1, 1)))
      model.add(Activation('relu'))
      model.add(Flatten())
      model.add(Dense(512))
      model.add(Activation('relu'))
      model.add(Dense(nb_actions))
      model.add(Activation('linear'))
  
      return model  
\end{lstlisting}
  
  

\section{Data Serializer}

Data SerializerはPostgreSQLサーバー上に記録された経路などの地理データーを上述したDeep Q Neural Network
に学習させるためのデーター変換を行うプログラムである.
開発言語はPythonである. PostgreSQLから受け取った道路のPOLYLINEデータから各座標点を抽出し, 座上間を繋いだ



\begin{figure}[H]
  \centering  % 図を真ん中に配置
  \includegraphics[clip,width = 13.0cm]{assets/map2vector.eps}
  \caption{地理データをDQNが学習可能な正方行列データに変換する}  \label{map2vector}
\end{figure}



\section{学習用コンテナの作成}

Docker\footnote{DockerはOS仮想化システムの一種である. VirtualBoxなどの完全仮想化
ソフトウェアとは異なり、ゲストOSの命令セットをホストOSのカーネルの命令セットにコンバートすることにより仮想化を実現している.
これにより, ハードウェアの仮想化を伴わないためオーバーヘッドが少なく機械学習のような計算量の多い課題に適していると言える.}
本研究では, 機械学習システムの実行環境として採用をした.
-----
主に, Pythonの実行環境系やライブラリなどをUbuntu18.04ベースのコンテナを作成した.
このコンテナには本研究で用いた機械学習システムが依存するTensorFlowの実行環境が用意されている.

\section{本研究で用いる環境定義アルゴリズム}

ここにアルゴリズムを示す

\textbf{for} Routelist

\ \ \ \ 評価関数

\textbf{end for}


\section{環境定義クラス}

\begin{lstlisting}[caption = 環境を構築するクラス, label = program1]

class CATPEnvironment:

  def __init__(self, evaluator, step_t):
      self._evaluator = evaluator
      self._step = step_t
      self._rewords = [ ]

  # Register or memory agent action flow.
  def act(self, act_obj):
      res = act_obj.run()
      rewords = self._evaluator(res)

      self._rewords.append(rewords)
  
  # Calculate environment diff and save environment.
  def commit(self):
      return sum(self._rewords)
  
  # Clear all environment to initial.
  def reset(self):
      total_reword = sum(self._rewords)

  def _capacity_validate(self, route_id):
      if True:
          return -10
\end{lstlisting}
  




\begin{lstlisting}[caption = 行動を定義するクラス, label = program1]

class CATPAction:

  def __init__(self):
    pass

  def reset(self):
    print("Reset environment")

  # Register agent action to environment
  def act(self, act_obj):
    pass

  # Calculate environment diff
  def commit(self):
    print("Committing environment...")

  def _capacity_validate(self, route_id):  
    if True:
        return -10  
\end{lstlisting}
  


%%% Local Variables:
%%% mode: japanese-latex
%%% TeX-master: "../bthesis"
%%% End:
