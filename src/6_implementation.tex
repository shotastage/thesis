\chapter{実験}
\label{implementation}

本章では提案手法の実験について述べる.

\section{概要}

本研究では, 以下に述べるコンピュータープログラムを実装し, モビリティや人の動きをコンピューター上で再現する.
作成するプログラムは 第一は, 強化学習モデルを生成する機械学習プログラムと

\section{構成}

この実験での各プログラムやミドルウェアの関係性をいかに示す.


\begin{figure}[H]
  \centering  % 図を真ん中に配置
  \includegraphics[clip,width = 13.0cm]{assets/system_overview.eps}
  \caption{システムの関係図}  \label{sample}
\end{figure}

\section{Deep Q Neural Network Model}

このプログラムはDeep Q Nueral Networkによる強化学習を行うプログラムである. 実装言語はPythonである.



\section{Data Serializer}

Data SerializerはPostgreSQLサーバー上に記録された経路などの地理データーを上述したDeep Q Neural Network
に学習させるためのデーター変換を行うプログラムである.
開発言語はPythonであり, 一部データーベースとの接続部分にC++を用いた.


\section{可視化地図アプリケーション}

モビリティの動きを地図上に可視化するアプリケーションの作成を行った. このアプリケーションはDeep Q Neural Networkの学習済モデルが
下した意思決定つまり経路を地図上にアイコンとして再現し, 乗車している人間に関しての情報を統合して閲覧できるようにしたものである.


\begin{table}[h]
  \caption{アプリケーション作成に用いたソフトウェアと詳細}
  \label{table:SpeedOfLight}
  \centering
  \begin{tabular}{clll}
    \hline
      使用対象 & 使用言語 & フレームワーク/ミドルウェア & OSなど \\
      \hline \hline
      可視化アプリケーション & Swift 5.1.3 & UIKit, SwiftUI & iOS \\
      可視化アプリケーション & JavaScript & ReactNative, JavaScriptCore & iOS \\
      GIS化データ管理 & pSQL & PostgreSQL 12.1 + PostGIS 3.0 & CoreOS \\
    \hline
  \end{tabular}
\end{table}


\section{Docker}

DockerはOS仮想化システムの一種である. VirtualBoxやVMWare Workstationなどの完全仮想化
ソフトウェアとは異なり、ゲストOSの命令セットをホストOSのカーネルの命令セットにコンバートすることにより仮想化を実現している.
これにより, ハードウェアの仮想化を伴わないためオーバーヘッドが少なく機械学習のような計算量の多い課題に適していると言える.
本研究では, 機械学習システムの実行環境として採用をした.
-----
主に, Pythonの実行環境系やライブラリなどをUbuntu18.04ベースのコンテナを作成した.
このコンテナには本研究で用いた機械学習システムが依存するTensorFlowの実行環境が用意されている.

\section{本研究で用いる環境定義アルゴリズム}

ここにアルゴリズムを示す

\textbf{for} Routelist

\ \ \ \ 評価関数

\textbf{end for}


\section{環境定義クラス}

\begin{lstlisting}[caption = 環境を構築するクラス, label = program1]


  class CATPEnvironment:

  def __init__(self, evaluator, step_t):
      self._evaluator = evaluator
      self._step = step_t
      self._rewords = [ ]

  # Register or memory agent action flow.
  def act(self, act_obj):
      res = act_obj.run()
      rewords = self._evaluator(res)

      self._rewords.append(rewords)
  
  # Calculate environment diff and save environment.
  def commit(self):
      return sum(self._rewords)
  
  # Clear all environment to initial.
  def reset(self):
      total_reword = sum(self._rewords)

  def _capacity_validate(self, route_id):
      if True:
          return -10 
\end{lstlisting}
  


\section{本実験における前提}

本実験では道路交通網をソフトウェアプログラムにより擬似的に再現する. その際に、実験を短略化させ結果が発生する要因を掴みやすくするため以下の条件を適用する.


\begin{table}[h]
  \caption{本実験で適用する条件}
  \label{table:SpeedOfLight}
  \centering
  \begin{tabular}{clll}
    \hline
      条件 & 理由 \\
      \hline \hline
      全ての道路は上下線通行可能 & 片道通行は複雑性が増すため  \\
      通行止めは考えない & 事故や災害などのアクシデント対応は \\ 本研究の趣旨と合わないため \\
    \hline
  \end{tabular}
\end{table}



\begin{itemize}
  \item 全ての道路は上下線双方向の通行が可能とする.
  \item 通行止めの発生は考えないものとする.
  \item 信号による制御などは考えないものとする.
\end{itemize}

%%% Local Variables:
%%% mode: japanese-latex
%%% TeX-master: "../bthesis"
%%% End:
