\chapter{実験}
\label{implementation}

本章では提案手法の実験について述べる.

\section{概要}

本研究では, 以下に述べるコンピュータープログラムを作成し, モビリティや利用者の満足度の変化をコンピューター上で再現する.
作成するプログラムは深層強化学習を行う学習器と環境定義プログラムである.

\section{構成}

この実験での各プログラムやミドルウェアの関係性をいかに示す.


\begin{figure}[H]
  \centering  % 図を真ん中に配置
  \includegraphics[clip,width = 13.0cm]{assets/system_overview.eps}
  \caption{システムの関係図}  \label{sample}
\end{figure}




\begin{table}[h]
  \caption{実験に使用したソフトウェア一覧と役割}
  \label{table:SpeedOfLight}
  \centering
  \begin{tabular}{clll}
    \hline
      名称 & 役割 \\
      \hline \hline
      PostgreSQL & 道路データの管理 \\
      PostGIS & 地理データの表現と格納 \\
      pgAdmin & 地理データのビジュアライズ \\
      QGIS & Open Street Mapの情報読み込み及地理データの閲覧 \\
    \hline
  \end{tabular}
\end{table}


\subsection{仮装道路モデルの作成}

本研究では, 実際の道路を参考にいくつかの道路を抜き出した仮装の道路網を作成した.
選択した道路の線形はPostgreSQLにてデータ化を行った. なお, この実験で使用したPostgreSQLには地理情報を扱う拡張であるPostGISをインストールしている.
データ化した内容は, 道路網の一定間隔の座標である. この座標同士を結ぶ線分をPOLYLINE型で記憶し路線名や道路が受け入れることのできる車の第数=Capacityを定義した.

\begin{lstlisting}[caption = 路線データを表すクエリーの例, label = program1]

/*
    テーブル構造の定義
*/
CREATE TABLE geodb_catp_yokohama (
    id                  serial PRIMARY KEY,
    route_name          VARCHAR (100),
    has_spot    VARCHAR (20),
    capacity            INTEGER
);

/*
    線地理情報を記録するPOLYLINE型のカラムを追加
*/
SELECT AddGeometryColumn ('public', 'geodb_catp_yokohama', 'geometry_data', 4326, 'LINESTRING', 2);

/*
    道路の線形を表すデータのレコード
*/
INSERT INTO geodb_catp_yokohama (route_name, has_spot, capacity, geometry_data)
VALUES ('K1',
        'none',
        200,
        ST_GeomFromText('LINESTRING(139.633639 35.445911, 139.632631 35.447126 ...... 139.632323 35.472085)', 4326));
\end{lstlisting}



\section{Deep Q Neural Network Model}

この実験ではDeep Q Nueral Networkによる強化学習を行う学習器を構築するプログラムをPython言語を用いて作成した.
このプログラムは深層強化学習を形成するニューラルネットワークを定義したプログラムである.
ライブラリとしてTensorFlow及びKerasを用いた.


\begin{lstlisting}[caption = DQNの深層学習部分を形成するモデル, label = program1]
  from tensorflow.keras.models import Sequential
  from tensorflow.keras.layers import (
    Dense,
    Activation,
    Flatten,
    Convolution2D,
    Permute
  )
  from tensorflow.keras.optimizers import Adam
  import tensorflow.keras.backend as K
  
  
  def dqnmodel():
      model = Sequential()
  
      # (width, height, channels)
      model.add(Permute((2, 3, 1), input_shape=input_shape))
  
      model.add(Convolution2D(32, (8, 8), strides=(4, 4)))
      model.add(Activation('relu'))
      model.add(Convolution2D(64, (4, 4), strides=(2, 2)))
      model.add(Activation('relu'))
      model.add(Convolution2D(64, (3, 3), strides=(1, 1)))
      model.add(Activation('relu'))
      model.add(Flatten())
      model.add(Dense(512))
      model.add(Activation('relu'))
      model.add(Dense(nb_actions))
      model.add(Activation('linear'))
  
      return model  
\end{lstlisting}
  
  

\section{Data Serializer}

Data SerializerはPostgreSQLサーバー上に記録された経路などの地理データーを上述したDeep Q Neural Network
に学習させるためのデーター変換を行うプログラムである.
開発言語はPythonであり, 一部データーベースとの接続部分にC++を用いた.

選び出したルートの正規化を行った。実際の地図上にある道路のPOLYLINEデータはそのままではDQNに学習させることができない。そこで、本研究では、1つのPOLYLINEを1つの

\section{可視化地図アプリケーション}

モビリティの動きを地図上に可視化するアプリケーションの作成を行った. このアプリケーションはDeep Q Neural Networkの学習済モデルが
下した意思決定つまり経路を地図上にアイコンとして再現し, 乗車している人間に関しての情報を統合して閲覧できるようにしたものである.


\begin{table}[h]
  \caption{アプリケーション作成に用いたソフトウェアと詳細}
  \label{table:SpeedOfLight}
  \centering
  \begin{tabular}{clll}
    \hline
      使用対象 & 使用言語 & フレームワーク/ミドルウェア & OSなど \\
      \hline \hline
      可視化アプリケーション & Swift 5.1.3 & UIKit, SwiftUI & iOS \\
      可視化アプリケーション & JavaScript & ReactNative, JavaScriptCore & iOS \\
      GIS化データ管理 & pSQL & PostgreSQL 12.1 + PostGIS 3.0 & CoreOS \\
    \hline
  \end{tabular}
\end{table}


\section{Docker - 確かにDockerあんまいらん}

Docker\footnote{DockerはOS仮想化システムの一種である. VirtualBoxなどの完全仮想化
ソフトウェアとは異なり、ゲストOSの命令セットをホストOSのカーネルの命令セットにコンバートすることにより仮想化を実現している.
これにより, ハードウェアの仮想化を伴わないためオーバーヘッドが少なく機械学習のような計算量の多い課題に適していると言える.}
本研究では, 機械学習システムの実行環境として採用をした.
-----
主に, Pythonの実行環境系やライブラリなどをUbuntu18.04ベースのコンテナを作成した.
このコンテナには本研究で用いた機械学習システムが依存するTensorFlowの実行環境が用意されている.

\section{本研究で用いる環境定義アルゴリズム}

ここにアルゴリズムを示す

\textbf{for} Routelist

\ \ \ \ 評価関数

\textbf{end for}


\section{環境定義クラス}

\begin{lstlisting}[caption = 環境を構築するクラス, label = program1]

class CATPEnvironment:

  def __init__(self, evaluator, step_t):
      self._evaluator = evaluator
      self._step = step_t
      self._rewords = [ ]

  # Register or memory agent action flow.
  def act(self, act_obj):
      res = act_obj.run()
      rewords = self._evaluator(res)

      self._rewords.append(rewords)
  
  # Calculate environment diff and save environment.
  def commit(self):
      return sum(self._rewords)
  
  # Clear all environment to initial.
  def reset(self):
      total_reword = sum(self._rewords)

  def _capacity_validate(self, route_id):
      if True:
          return -10
\end{lstlisting}
  




\begin{lstlisting}[caption = 行動を定義するクラス, label = program1]

class CATPAction:

  def __init__(self):
    pass

  def reset(self):
    print("Reset environment")

  # Register agent action to environment
  def act(self, act_obj):
    pass

  # Calculate environment diff
  def commit(self):
    print("Committing environment...")

  def _capacity_validate(self, route_id):  
    if True:
        return -10  
\end{lstlisting}
  
  



\section{本実験における前提}

本実験では道路交通網をソフトウェアプログラムにより擬似的に再現する. その際に、実験を短略化させ結果が発生する要因を掴みやすくするため以下の条件を適用する.


\begin{table}[h]
  \caption{本実験で適用する条件}
  \label{table:SpeedOfLight}
  \centering
  \begin{tabular}{clll}
    \hline
      条件 & 理由 \\
      \hline \hline
      全ての道路は上下線通行可能 & 片道通行は複雑性が増すため  \\
      通行止めは考えない & 事故や災害などのアクシデント対応は \\ 本研究の趣旨と合わないため \\
    \hline
  \end{tabular}
\end{table}



\begin{itemize}
  \item 全ての道路は上下線双方向の通行が可能とする.
  \item 通行止めの発生は考えないものとする.
  \item 信号による制御などは考えないものとする.
\end{itemize}

%%% Local Variables:
%%% mode: japanese-latex
%%% TeX-master: "../bthesis"
%%% End:
