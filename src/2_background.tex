\chapter{背景}
\label{background}



本章では, 本研究の背景を示す.

\subsection{自動運転}

車における自動運転は1980年代から研究されてきた.
例えば, 欧州で1987年から1995年に行われたEUREKAプロメテウス計画\footnote[1]{https://www.eurekanetwork.org/project/id/45}では高速道路における車線の追従や車線の変更などの自動運転の基礎技術が研究された.
現在では, これらの機能は市販の自家用車にも運転を支援する機能として搭載されている. また, 高速道路など限定した場所であれば人間による介入が不要な一部自動運転が可能となっているものもある.
今後, 将来自動運転技術はより人間による操作を少なくし, 首相官邸ホームページ「官民 ITS 構想・ロードマップ 2019」\footnote[2]{https://www.kantei.go.jp/jp/singi/it2/kettei/pdf/20190607/siryou9.pdf}に定義されたレベル5の完全な自動運転技術が完成する可能性がある.



\subsection{Mobility as a Service}

日本に置いて, 車や鉄道などの交通は高度経済成長期以降, 道路網や路線網の拡大も合わせて急速に普及が進み, 旅客・貨物共に主たる移動手段となった. 

しかし, 近年, 交通は単なる移動手段としてだけではなく, 移動や移動に付随する付加価値や自己所有の車を自ら運転すると行った従来の使い方からの変化が求められるようになってきた.

これに対して移動をサービスとして提供するという考え方があり, Mobility as a Serviceの略称MaaSと呼ばれている. 現在MaaSサービスとしては米Uberなどに代表される個人所有の車を配車するサービスや, 自動車を不特定多数の利用者で共有するカーシェアリングサービスなどがある.

人間による操作を必要としないレベル5の自動運転が実現すると, ハンドルを握る必要がないため従来のように自動車において移動時間中に運転に拘束されることがない.
また, 移動経路も人間が考えることなく目的地まで到達する. このようになると, 単に自動車そのものを共有するサービスだけではなく
移動経路の選択や移動時間の活用などのサービスとして提供する事が必要になると予想される.

\subsection{シェアリングエコノミー}

シェアリングエコノミー\cite{Sharing}ないしは共有経済とはモノやサービスを特定の個人で所有するのではなく複数人で共有する社会関係もしくは社会システムである.
古くはGNUプロジェクトなどのオープンソースソフトウェアがそれに当たると考えられている.
スマートフォンの普及により人々は常時インターネットに接続するようになった, これによりUber\footnote[1]{https://www.uber.com/jp/ja/}などの個人所有の車の配車サービスやAirBnb\footnote[2]{https://www.airbnb.jp}などの所有する不動産を一定期間旅館のような形で貸し出すと行ったサービスが登場し,
社会的に大きな影響をもたらしている.
将来, 人間が介在することのない自動運転が可能になると自動車そのもののハードウェアをシェアするだけではなく自動車を使ったサービスのシェアが加速すると思われる.


\subsection{機械学習}

機械学習とはコンピューターが自動的にパターンを学習し人間による指示がなくとも課題を実行する事である.
モバイルコンピューターの性能が飛躍的に向上し画像の認識や機械翻訳などの多くの分野で応用されるようになってきた.
これらの技術は日々進化を重ねており画像分類の分野においては1年で認識率を20\%向上したり, 学習にかかる時間を50\%ほど短縮するアルゴリズムが考案されたりと飛躍的に向上してる.
また, 機械学習アルゴリズムは意思決定を伴う処理にも活用され, AlphaGo~\cite{AlphaGo}に代表されるように囲碁などのボードゲーム分野におて人間をも上回る成果を出している.
加えて, 意思決定型の機械学習はボードゲーム意外にも応用されつつあり, リスティング広告の最適化など利用者である人間にとっての利益が最大化するような制御に応用されつつある.


%%% Local Variables:
%%% mode: japanese-latex
%%% TeX-master: "../thesis"
%%% End:
