\chapter{背景}
\label{background}


\section{自動運転}

車における自動運転は1980年代から研究されてきた.
例えば, 欧州で1987年から1995年に行われたEUREKAプロメテウス計画では...などの自動運転の基礎技術が研究された.
現在では, 


\section{Mobility as a Service}

日本に置いて, 車や鉄道などの交通は高度経済成長期以降, 急速に普及が進み, 旅客・貨物共に主たる移動手段となった. 

しかし, 近年, 交通は単なる移動手段としてだけではなく, 移動や移動に付随する付加価値が求められるようになってきた.

これに対して移動をサービスとして提供しようという試みがあり, Mobility as a Service (通称: MaaS)と呼ばれている.


\section{シェアリングエコノミー}

シェアリングエコノミーとは....である. 将来, 人間が介在することのない自動運転が可能になると

\section{機械学習}

日本に置いて高度経済成長期以降, 車や鉄道の普及は急速に進み旅客・貨物共に主たる移動手段となった.

しかし, 近年, 交通は単なる移動手段としてだけではなく, 移動や移動に付随する付加価値が求められるようになってきた.
これに対して移動をサービスとして提供しようという試みがあり, Mobility as a Service (通称: MaaS)と呼ばれている.


\section{5G}

日本に置いて高度経済成長期以降, 車や鉄道の普及は急速に進み旅客・貨物共に主たる移動手段となった.

しかし, 近年, 交通は単なる移動手段としてだけではなく, 移動や移動に付随する付加価値が求められるようになってきた.
これに対して移動をサービスとして提供しようという試みがあり, Mobility as a Service (通称: MaaS)と呼ばれている.


\if0
\begin{figure}[h]
    \begin{center}
        \includegraphics[scale=0.4]{./img/hashrate.png}
        \caption{2017年1月のハッシュレート分布 出典:Blockchain.info\cite{bitcoinhashrate}}
        \label{img:hashrate}
    \end{center}
\end{figure}
\fi
