\chapter{実験}
\label{implementation}

本章では提案手法の実験について述べる.

\section{概要}

本研究では, 以下に述べるコンピュータープログラムを実装し, モビリティや人の動きをコンピューター上で再現する.
作成するプログラムは 第一は, 強化学習モデルを生成する機械学習プログラムと

\section{構成}

この実験での各プログラムやミドルウェアの関係性をいかに示す.


\begin{figure}[H]
  \centering  % 図を真ん中に配置
  \includegraphics[clip,width = 13.0cm]{assets/system_overview.eps}
  \caption{システムの関係図}  \label{system_overview}
\end{figure}

\section{Deep Q Neural Network Model}

このプログラムはDeep Q Nueral Networkによる強化学習を行うプログラムである. 実装言語はPythonである.



\section{Data Serializer}

Data SerializerはPostgreSQLサーバー上に記録された経路などの地理データーを上述したDeep Q Neural Network
に学習させるためのデーター変換を行うプログラムである.
開発言語はPythonであり, 一部データーベースとの接続部分にC++を用いた.


\section{可視化地図アプリケーション}

モビリティの動きを地図上に可視化するアプリケーションの作成を行った. このアプリケーションはDeep Q Neural Networkの学習済モデルが
下した意思決定つまり経路を地図上にアイコンとして再現し, 乗車している人間に関しての情報を統合して閲覧できるようにしたものである.


\begin{table}[h]
  \caption{光速度の測定の歴史}
  \label{table:SpeedOfLight}
  \centering
   \begin{tabular}{clll}
    \hline
    西暦 & 測定者 & 測定方法 & 測定結果 \\
     & & & $\times 10^8$ [m/sec] \\
    \hline \hline
    1638 & Galileo & 二人が離れてランプの光を見る & (音速10倍以上) \\
    1675 & Roemer & 木製の衛星の観測から & 2 \\
    1728 & Bradley & 星の収差から & 3.01 \\
    1849 & Fizeau & 高速に回転する歯車を通過する光を見る & 3.133 \\
    1862 & Foucault & 高速に回転する鏡の光の角度変化 & 2.99796 \\
    今日 & (定義) & & 2.99792458 \\
    \hline
   \end{tabular}
 \end{table}
 
\begin{table}[htb]
    \centering
    \begin{tabular}{|l|c|r||r|} \hline 
      ライブラリ名 / 言語など & バージョン & 実行デバイス \\ \hline \hline
      Swift & 5.1.3 (swiftlang-1100.0.282.1 clang-1100.0.33.15) & iPad Pro \\ \hline
      React Native & 0.61.5 & iPad Pro  \\ \hline
    \end{tabular}
\end{table}

%%% Local Variables:
%%% mode: japanese-latex
%%% TeX-master: "../bthesis"
%%% End:
