\chapter{考察}
\label{consideration}

自動車などのモビリティの経路選択においては, 最短経路を選択することで短時間で到達することが最適だとされてきた.
最近では, 米Google社のGoogle Map~\cite{googlemap}のように各端末から収集した位置情報を基に渋滞を予測し予想時間を計算することで
時間最短距離を求めるなど単に最短距離だけでなく, より短時間で目的地に到達できるようになった.
本研究では, 最短距離や時間最短距離に加えモビリティの利用者である人間の満足度に焦点をおいて最適化手法を提案し実験を行った.
結果, 最短距離ではなく満足度の高いルートを選択する傾向が見られ, 本研究の手法が人間の満足度を高める最適化においての一定の有効性が示せた.

また, 経路検索に機械学習を用いた事例は, あまり見られない.
なぜなら, 最短経路の検索では深層学習などの複雑な機械学習手法よりもダイクストラ法などの既存のアルゴリズムの方が計算量も少なく単純だからである.
しかし, 人間の満足度を踏まえた最適化では, 単純なアルゴリズムでは有効でない. 本研究での機械学習手法を最短経路検索に応用する余地がある.
