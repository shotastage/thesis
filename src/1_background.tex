\chapter{序論}
\label{background}

本章では本研究の背景,課題及び手法を提示し,本研究の概要を示す.

\section{背景}

あああああああああ


\section{自動運転}

車における自動運転は1980年代から研究されてきた.
例えば, 欧州で1987年から1995年に行われたEUREKAプロメテウス計画では高速道路における車線の追従や車線の変更などの自動運転の基礎技術が研究された.
現在では, これらの機能は市販の自家用車にも運転をアシストする機能として搭載されている. また, 高速道路など限定した場所であれば人間による介入が不要な一部自動運転が可能となっているものもある.
今後, 将来自動運転技術はより人間による介入を少なくし, 首相官邸ホームページ「官民 ITS 構想・ロードマップ 2017」に定義されたレベル5の完全な自動運転技術も完成する可能性がある.


\section{Mobility as a Service}

日本に置いて, 車や鉄道などの交通は高度経済成長期以降, 急速に普及が進み, 旅客・貨物共に主たる移動手段となった. 

しかし, 近年, 交通は単なる移動手段としてだけではなく, 移動や移動に付随する付加価値が求められるようになってきた.

これに対して移動をサービスとして提供しようという試みがあり, Mobility as a Service (通称: MaaS)と呼ばれている.


\section{シェアリングエコノミー}

[ここは削ると思う]

シェアリングエコノミーとは....である. 将来, 人間が介在することのない自動運転が可能になると

\section{機械学習}

日本に置いて高度経済成長期以降, 車や鉄道の普及は急速に進み旅客・貨物共に主たる移動手段となった.

しかし, 近年, 交通は単なる移動手段としてだけではなく, 移動や移動に付随する付加価値が求められるようになってきた.
これに対して移動をサービスとして提供しようという試みがあり, Mobility as a Service (通称: MaaS)と呼ばれている.


\section{5G}

日本に置いて高度経済成長期以降, 車や鉄道の普及は急速に進み旅客・貨物共に主たる移動手段となった.

しかし, 近年, 交通は単なる移動手段としてだけではなく, 移動や移動に付随する付加価値が求められるようになってきた.
これに対して移動をサービスとして提供しようという試みがあり, Mobility as a Service (通称: MaaS)と呼ばれている.




%%% Local Variables:
%%% mode: japanese-latex
%%% TeX-master: "../thesis"
%%% End:
