\chapter{序論}
\label{introduction}

本章では本研究の背景,課題及び手法を提示し,本研究の概要を示す.


\section{背景}
\label{introduction:background}

本章では, 本研究の背景を示す.

\subsection{自動運転}

車における自動運転は1980年代から研究されてきた.
例えば, 欧州で1987年から1995年に行われたEUREKAプロメテウス計画では高速道路における車線の追従や車線の変更などの自動運転の基礎技術が研究された.
現在では, これらの機能は市販の自家用車にも運転をアシストする機能として搭載されている. また, 高速道路など限定した場所であれば人間による介入が不要な一部自動運転が可能となっているものもある.
今後, 将来自動運転技術はより人間による介入を少なくし, 首相官邸ホームページ「官民 ITS 構想・ロードマップ 2017」に定義されたレベル5の完全な自動運転技術も完成する可能性がある.


\subsection{Mobility as a Service}

日本に置いて, 車や鉄道などの交通は高度経済成長期以降, 急速に普及が進み, 旅客・貨物共に主たる移動手段となった. 

しかし, 近年, 交通は単なる移動手段としてだけではなく, 移動や移動に付随する付加価値が求められるようになってきた.

これに対して移動をサービスとして提供しようという試みがあり, Mobility as a Service (通称: MaaS)と呼ばれている.

\subsection{シェアリングエコノミー}

ここは削ると思う
シェアリングエコノミーとは....である. 将来, 人間が介在することのない自動運転が可能になると

\subsection{機械学習}

日本に置いて高度経済成長期以降, 車や鉄道の普及は急速に進み旅客・貨物共に主たる移動手段となった.

しかし, 近年, 交通は単なる移動手段としてだけではなく, 移動や移動に付随する付加価値が求められるようになってきた.
これに対して移動をサービスとして提供しようという試みがあり, Mobility as a Service (通称: MaaS)と呼ばれている.


\subsection{5G}

日本に置いて高度経済成長期以降, 車や鉄道の普及は急速に進み旅客・貨物共に主たる移動手段となった.

しかし, 近年, 交通は単なる移動手段としてだけではなく, 移動や移動に付随する付加価値が求められるようになってきた.
これに対して移動をサービスとして提供しようという試みがあり, Mobility as a Service (通称: MaaS)と呼ばれている.






なお,Bitcoin~\cite{Bitcoin}は関係ない.



\section{本研究の目的}

本研究では, 強化学習を用いたルート選択を試行し人間の満足度を高めるモビリティ制御を目指す.
現状, 自動運転車に代表されるモビリティは個別に行動しており, ルートの選択に置いても人間が与えた目的地への最短経路を選択している.
この場合, 一定時間内に同一方面の目的地を多数設定された場合に特定のルートの混雑が予測される.
本研究では, 



\section{本論文の構成}

本論文における以降の構成は次の通りである.

~\ref{introduction:background}章では,背景を述べる.
~\ref{issue}章では,本研究における問題の定義と,解決するための要件の整理を行う.
~\ref{proposed}章では,本研究の提案手法を述べる.
~\ref{implementation}章では,~\ref{proposed}章で述べたシステムの実装について述べる.
~\ref{evaluation}章では,\ref{issue}章で求められた課題に対しての評価を行い,考察する.
~\ref{conclusion}章では,本研究のまとめと今後の課題についてまとめる.


%%% Local Variables:
%%% mode: japanese-latex
%%% TeX-master: "../thesis"
%%% End: