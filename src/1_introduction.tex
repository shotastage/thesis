\chapter{序論}
\label{introduction}

本章では本研究の動機,本研究の概要を示す.

\textcolor{red}{Abstractからポート}
自動運転技術の発展により, 我々人間は車を常時コントロールする必要がなくなってきた.
例えば, 車線の自動変更や物体検知による自動ブレーキなどは, 今まで人間が行っていた操作をコンピューターによって行っている.
今後, 自動運転技術はより高度になり, 近い将来人間がハンドルを握らなくともコンピューターによる制御のみで目的地まで到達できるレベルまで発展する可能性がある.
そのような自動運転車が普及した社会において, 自動車各個が個別に行動すると様々な問題が発生する.

第一に, 特定の経路の混雑があげられる.
多数の自動運転車が個別に経路を選択した場合, 特定の道が混雑する問題が発生する.
第二に, 自動運転による使用用途の変化への順応である. 人間による操作が一切行われない完全自動運転が実現する事はすなわち無人運転が可能であることを意味する.
無人運転が可能となった場合, 自動車の保有者が目的地についた後に駐車する必要はなく, 駐車時間中に自動車を使いたい人の元へ迎えに行く無人のヒッチハイクのような行為が可能となる.
このような使用用途の変化に対応するには, 単に目的地までのルートを選択するだけではなく目的地に到達した後の行動も自動運転車が決定する必要がある.




\section{本研究の目的}

将来の完全自動運転車が実現するにあたり, \textcolor{red}{加筆・修正}
本研究では, 強化学習を用いた経路選択を試行し自動車の利用者の移動目的をより多く達成できる強化学習モデルの構築を目指す.
現状, 自動運転車に代表されるモビリティは個別に行動しており, ルートの選択に置いても人間が与えた目的地への最短経路を選択している.
この場合, 一定時間内に同一方面の目的地を多数設定された場合に特定のルートの混雑が予測される. 
また, レベル5の完全自動運転が実現された場合, 自動車を使用してない期間は自動で他者に貸し出すなどの利用方法の変化も見込まれる.

本研究では, 機械学習モデルの一つである深層強化学習を用いて, 経路選択の実験を行い, 自律意思決定型の機械学習アルゴリズムが自動運転車において活用可能であることを示したい.


\section{本論文の構成}

本論文における以降の構成は次の通りである.

~\ref{background}章では,背景を述べる.
~\ref{issue}章では,本研究における問題の定義と,解決するための要件の整理を行う.
~\ref{proposed}章では,本研究の提案手法を述べる.
~\ref{implementation}章では,~\ref{proposed}章で述べたシステムの実装について述べる.
~\ref{evaluation}章では,\ref{issue}章で求められた課題に対しての評価を行い,考察する.
~\ref{conclusion}章では,本研究のまとめと今後の課題についてまとめる.


%%% Local Variables:
%%% mode: japanese-latex
%%% TeX-master: "../thesis"
%%% End:
