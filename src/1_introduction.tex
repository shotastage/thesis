\chapter{序論}
\label{introduction}

本章では本研究の背景,課題及び手法を提示し,本研究の概要を示す.

\section{はじめに}
\label{introduction:background}
慶應義塾大学SFCでは,卒業要件として卒業論文の執筆が必要とされている.
近年,多くの学生が提出間近になってから卒論を執筆することが多くなっている.
そうした学生の多くは,残留を繰り返し,魔剤を飲みながらデスレースを実施することとなる.

その中でも,\LaTeX の理解は執筆において不可欠であり避けられない.
しかしながら,多くの学生はWIP/TERMで予稿の執筆を怠り,いざ執筆を始めようとしても\LaTeX を用いて論文を執筆することが難しい.

そこで,本研究ではRGの学生に向けて心優しい博士課程として,RGの卒業論文のスタイルに合った形であると言われているテンプレートを整理し,提供する.
本テンプレートでは,基本的な章立ての中で,\LaTeX の使い方を概説し,このクソみたいな文章を削除し,卒業論文を執筆するにあたって基本的な記法を理解できることを期待する.

なお,Bitcoin~\cite{Bitcoin}は関係ない.

\section{本論文の構成}

本論文における以降の構成は次の通りである.

~\ref{background}章では,背景を述べる.
~\ref{issue}章では,本研究における問題の定義と,解決するための要件の整理を行う.
~\ref{proposed}章では,本研究の提案手法を述べる.
~\ref{implementation}章では,~\ref{proposed}章で述べたシステムの実装について述べる.
~\ref{evaluation}章では,\ref{issue}章で求められた課題に対しての評価を行い,考察する.
~\ref{conclusion}章では,本研究のまとめと今後の課題についてまとめる.


%%% Local Variables:
%%% mode: japanese-latex
%%% TeX-master: "../thesis"
%%% End:
