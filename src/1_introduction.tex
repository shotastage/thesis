\chapter{序論}
\label{introduction}

本章では本研究の動機,本研究の概要を示す.

\section{本研究の目的}

本研究では, 強化学習を用いた経路選択を試行し自動車の利用者の移動目的をより多く達成できる強化学習モデルの構築を目指す.
現状, 自動運転車に代表されるモビリティは個別に行動しており, ルートの選択に置いても人間が与えた目的地への最短経路を選択している.
この場合, 一定時間内に同一方面の目的地を多数設定された場合に特定のルートの混雑が予測される. 
また, レベル5の完全自動運転が実現された場合, 自動車を使用してない期間は自動で他者に貸し出すなどの利用方法の変化も見込まれる.

本研究では, 機械学習モデルの一つである深層強化学習を用いて, 経路選択の実験を行い, 自立意思決定型の機械学習アルゴリズムが自動運転車において活用可能であることを示したい.




\section{本論文の構成}

本論文における以降の構成は次の通りである.

~\ref{background}章では,背景を述べる.
~\ref{issue}章では,本研究における問題の定義と,解決するための要件の整理を行う.
~\ref{proposed}章では,本研究の提案手法を述べる.
~\ref{implementation}章では,~\ref{proposed}章で述べたシステムの実装について述べる.
~\ref{evaluation}章では,\ref{issue}章で求められた課題に対しての評価を行い,考察する.
~\ref{conclusion}章では,本研究のまとめと今後の課題についてまとめる.


%%% Local Variables:
%%% mode: japanese-latex
%%% TeX-master: "../thesis"
%%% End: