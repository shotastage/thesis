\chapter*{謝辞}
\addcontentsline{toc}{chapter}{謝辞}
\label{thanks}

本論文の執筆にあたり,ご指導頂いた慶應義塾大学環境情報学部村井純博士,同学部教
授中村修博士,同学部教授楠本博之博士,同学部准教授高汐一紀博士,同学部教授三次仁
博士,同学部准教授植原啓介博士,同学部准教授中澤仁博士,同学部準教授 Rodney D.
Van Meter III 博士,同学部教授武田圭史博士,同大学政策・メディア研究科特任准教授
鈴木茂哉博士,同大学政策・メディア研究科特任准教授佐藤 雅明博士,同大学 SFC 研究
所上席所員斉藤賢爾博士に感謝致します.

特に斉藤氏には重ねて感謝致します.研究活動の中で様々な視点での助言をいただきました. 

政策・メディア研究科 阿部 涼介氏に重ねて感謝いたします. 私が同合同研究会に所属してから今日まで, 研究活動を支援し

続けて, 村井研究室NECOの松本 光生氏, 風間 宏治氏, 宮元 眺氏は長くNECOとしての生活を共にし,
日々の研究生活の助けとなりました. 彼らと過ごした日々は私の興味範囲を広げると共に多くの刺激を得ることができました. 深く感謝いたします.
また, 梶原 留衣氏, 渡辺 聡紀氏, 木内 啓介氏, 後藤 悠太氏, 倉重 健氏, 九鬼 嘉隆氏,
内田 渓太氏, 山本 哲平氏, 吉開 拓人氏, 金城 奈菜海氏, 長田 琉羽里氏, 前田 大輔氏には
NECOのグループリーダーである私を様々な面から支えて頂き心から感謝しています. 彼らの協力なしには研究グループの運営はできませんでした.

学外で私を様々な面で支えてくださった佐野 美月氏に深く感謝申し上げます.

私に, この卒論を書く学びの場を提供しこの世に生を受けてから今日まで私を支えてくださった. 両親並びに弟に深く感謝いたします.

最後に, 私をここまで成長させる基盤となった日本国並びに全地球に感謝いたします.


%%% Local Variables:
%%% mode: japanese-latex
%%% TeX-master: "../yummy_bthesis"
%%% End:
