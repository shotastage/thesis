\chapter*{謝辞}
\addcontentsline{toc}{chapter}{謝辞}
\label{thanks}

本論文の執筆にあたり,ご指導頂いた慶應義塾大学環境情報学部村井純博士,同学部教
授中村修博士,同学部教授楠本博之博士,同学部准教授高汐一紀博士,同学部教授三次仁
博士,同学部准教授植原啓介博士,同学部准教授中澤仁博士,同学部準教授 Rodney D.
Van Meter III 博士,同学部教授武田圭史博士,同大学政策・メディア研究科特任准教授
鈴木茂哉博士,同大学政策・メディア研究科特任准教授佐藤 雅明博士,同大学 SFC 研究
所上席所員斉藤賢爾博士に感謝致します.


本論文の執筆にあたり,ご指導頂いた慶應義塾大学環境情報学部村井純博士,同学部教 授中村修博士,同学部教授楠本博之博士
同学部准教授高汐一紀博士,同学部教授三次仁 博士,同学部准教授植原啓介博士,同学部准教授中澤仁博士,同学部準教授 Rodney D. Van Meter III 博士
同学部教授武田圭史博士,同大学政策・メディア研究科特任准教授 鈴木茂哉博士,同大学政策・メディア研究科特任准教授佐藤 雅明博士
同大学 SFC 研究 所上席所員斉藤賢爾博士に感謝致します.
特に斉藤氏には重ねて感謝致します.
研究活動を通して研究グループNECOがターゲットとする技術的視点, 社会的視点等の両側面から私の研究に対して助言を頂き,
私だけでは発想にも至らなかった様々な考えや知見を頂きました. 博士の指導なしには、卒業論文を執筆することは出来ませんでした.
徳田・村井・楠本・中村・高汐・バンミーター・植原・三次・中澤・武田合同研究プロ ジェクトに所属している学部生,大学院生,卒業生の皆様に感謝致します.
研究会に所属する多くの方々が各々の分野・研究で奮闘している姿を見て学んだことが私の研究生活をより充実したものとさせました.異なる分野同士が触れ合い,学び合う環境に出会えたことを嬉しく感じます.

慶應義塾大学 政策・メディア研究科 阿部 涼介氏に重ねて感謝いたします. 私が同合同研究会に所属してから今日まで, 研究活動を支援して頂きました.

続けて, 村井研究室NECOの卒業生 菅藤佑太氏, 在校生 松本 光生氏, 風間 宏治氏, 宮元 眺氏は長くNECOとしての生活を共にし,
日々の研究生活の助けとなりました. 彼らと過ごした日々は私の興味範囲を広げると共に多くの刺激を得ることができました. 深く感謝いたします.
また, 梶原 留衣氏, 渡辺 聡紀氏, 木内 啓介氏, 後藤 悠太氏, 倉重 健氏, 九鬼 嘉隆氏,
内田 渓太氏, 山本 哲平氏, 吉開 拓人氏, 金城 奈菜海氏, 長田 琉羽里氏, 前田 大輔氏には
NECOのグループリーダーである私を様々な面から支えて頂き心から感謝しています. 彼らの協力なしには研究グループの運営はできませんでした.

私に, この卒論を書く学びの場を提供しこの世に生を受けてから今日まで私を支えてくださった. 両親並びに弟に深く感謝いたします.

最後に, 私をここまで成長させる基盤となった日本国並びに全地球に感謝いたします.


%%% Local Variables:
%%% mode: japanese-latex
%%% TeX-master: "../yummy_bthesis"
%%% End:
