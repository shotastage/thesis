卒業論文要旨 - 2019年度 (令和1年度)
\begin{center}
\begin{large}
\begin{tabular}{|M{0.97\linewidth}|}
    \hline
      \title \\
    \hline
\end{tabular}
\end{large}
\end{center}

~ \\

自動運転技術の発展により, 我々人間は車を常時コントロールする必要がなくなってきた.
例えば, 車線の自動変更や物体検知による自動ブレーキなどは, 今まで人間が行っていた操作をコンピューターによって行っている.
今後, 自動運転技術はより高度になり, 近い将来人間がハンドルを握らなくともコンピューターによる制御のみで目的地まで到達できるレベルまで発展する可能性がある.
そのような自動運転車が普及した社会において, 自動車各個が個別に行動すると様々な問題が発生する. 第一に, 特定の経路の混雑があげられる.
多数の自動運転車が個別に経路を選択した場合, 特定の道が混雑する問題が発生する.
第二に, 自動車の所有スタイルの変化による使用用途の変化への順応である. 人間による操作が一切行われない完全自動運転が実現する事はすなわち無人運転が可能であることを意味する.
無人運転が可能となった場合, 自動車の保有者が目的地についた後に駐車する必要はなく, 駐車時間中に自動車を使いたい人の元へ迎えに行く無人のヒッチハイクのような行為が可能となる.
この場合, 目的地に時間最短を求める現状のルート選択手法では最適な解が得られない可能性が発生する.

そこで, 本研究では強化学習を用いて自動車を利用した人の目的にあったルート選択を行えるかを実験する. 具体的には, 



~ \\
キーワード:\\
\underline{1. 機械学習},
\underline{2. 強化学習},
\underline{3. モビリティ}
\begin{flushright}
\dept \\
\author
\end{flushright}
