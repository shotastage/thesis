卒業論文要旨 - 2019年度 (令和1年度)
\begin{center}
\begin{large}
\begin{tabular}{|M{0.97\linewidth}|}
    \hline
      \title \\
    \hline
\end{tabular}
\end{large}
\end{center}

~ \\

自動運転技術の発展により, 我々人間は車を常時コントロールする必要がなくなってきた.
例えば, 車線の自動変更や物体検知による自動ブレーキなどは, 今まで人間が行っていた操作をコンピューターによって行っている.
今後, 自動運転技術はより高度になり, 近い将来人間がハンドルを握らなくともコンピューターによる制御のみで目的地まで到達できるレベルまで発展する可能性がある.
そのような自動運転車が普及した社会において, 自動車各個が個別に行動すると様々な問題が発生する.

第一に, 特定の経路の混雑があげられる.
多数の自動運転車が個別に経路を選択した場合, 特定の道が混雑する問題が発生する.
第二に, 自動運転による使用用途の変化への順応である. 人間による操作が一切行われない完全自動運転が実現する事はすなわち無人運転が可能であることを意味する.
無人運転が可能となった場合, 自動車の保有者が目的地についた後に駐車する必要はなく, 駐車時間中に自動車を使いたい人の元へ迎えに行く無人のヒッチハイクのような行為が可能となる.
このような使用用途の変化に対応するには, 単に目的地までのルートを選択するだけではなく目的地に到達した後の行動も自動運転車が決定する必要がある.

そこで, 本研究では強化学習を用いて自動車の利用者の目的に合わせた経路選択の実験を行う.
実験では, 緊急のミーティングに向かっている, 空腹を感じているなどの利用者の目的や状態などのパターンを予め想定した. そのパターンに対して利用者の目的を達成するルートを選択した場合に強化学習に報酬値を与え, 目的を満たせないルートを選択した場合は罰則値を与える.
これを繰り返すことにより, 強化学習アルゴリズムがより目的を達成するような経路選択を行えるようになると考えられる.

これらの実験を通して, 本研究では正解データなしに自立して意思決定を行うアルゴリズムの一種である強化学習を用いたアプローチが人間の目的を満たす意思決定に有効であること. モビリティの経路選択において実用的であることを示す事を期待する.

~ \\
キーワード:\\
\underline{1. 機械学習},
\underline{2. 強化学習},
\underline{3. モビリティ}
\begin{flushright}
\dept \\
\author
\end{flushright}
