卒業論文要旨 - 2019年度 (令和1年度)
\begin{center}
\begin{large}
\begin{tabular}{|M{0.97\linewidth}|}
    \hline
      \title \\
    \hline
\end{tabular}
\end{large}
\end{center}

~ \\

自動車技術の発展により運転に関わる一部の行為をコンピューターで自動化されるようになった. 例えば, 車線の自動変更や物体検知による自動ブレーキなどである.
今後, 自動運転技術はより高度になり, 近い将来人間がハンドルを握らなくともコンピューターによる制御飲みで目的地まで到達できるレベルまで発展する可能性がある.
そうなった社会に置いて, 必ずしも個人が自動運転車を保有する必要はなくなり

そこで, 本研究では各車載AIをシミュレーター上で再現し互いに
コミュニケーション・連携をとることにより全体最適解を求める実験を行う.


~ \\
キーワード:\\
\underline{1. 機械学習},
\underline{2. 強化学習},
\underline{3. モビリティ}
\begin{flushright}
\dept \\
\author
\end{flushright}
