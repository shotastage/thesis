\chapter{本研究における問題定義と仮説}
\label{issue}

本章では, 第~\ref{background}章で述べた背景から, 現状の自動運転システムが目的としている制御方法の問題点を述べる.

\section{最短経路問題}

[MARK: あとで背景の章に移動]

自動運転が行われるようになると, 人間による非合理的なブレーキなどの操作がなくなり渋滞が解消すると言われている.
しかし, 渋滞は車線あたりの交通量に比例し, 一般的に自動車の走行ルートを決定する場合, 現在地から目的地までの最短ルートを選択するため
交通需要の高い経路の交通量は変わらず, 自動運転による渋滞解消の因果関係には疑問が残る.
少なくとも同時間帯に同一地点付近の目的地を設定した車が大量にいた場合に特定のルートが混雑が発生することは避けられないと考えられる.
現状のカーナビゲーションシステムなどに搭載されているような渋滞回避機能であっても

\section{強化学習によるルート選択}

ああああ


\section{ケース策定}

ああああああ

\section{Other priblems}

ああああああああああああ

%%% Local Variables:
%%% mode: japanese-latex
%%% TeX-master: "./thesis"
%%% End:
