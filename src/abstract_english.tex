Abstract of Bachelor's Thesis - Academic Year 2019
\begin{center}
\begin{large}
\begin{tabular}{|p{0.97\linewidth}|}
    \hline
      \etitle \\
    \hline
\end{tabular}
\end{large}
\end{center}

~ \\
With the development of autonomous driving technology, humans no longer need to constantly control cars.
For example, automatic lane changes and automatic braking based on object detection are done by computers in the same way that humans have done so far.
In the future, automatic driving technology will become more advanced, and there is a possibility that in the near future, it will be developed to a level where a person can reach a destination only by computer control without grasping the steering wheel.
In a society where such self-driving vehicles are widespread, various problems arise when individual vehicles act individually.

First, there is congestion on specific routes.
If a large number of autonomous vehicles select their own routes individually, there will be a problem of congestion on specific roads.
The second is to adapt to changes in usage by automatic operation. The realization of fully automatic operation without any human intervention means that unmanned operation is possible.
When unmanned driving becomes possible, it is not necessary to park a car after the car owner arrives at the destination, and an act such as an unmanned hitchhiking to pick up a person who wants to use the car during parking time becomes possible.
In order to respond to such changes in usage, it is necessary for the self-driving vehicle not only to select the route to the destination but also to determine the actions to take after reaching the destination.

In this study, therefore, we use reinforcement learning to experiment with route selection according to the purpose of vehicle users.
In the experiment, we assumed the user's purpose and condition such as going to an emergency meeting or feeling hungry. If the user selects a route that achieves the user's purpose for that pattern, we give a reward value to reinforcement learning. If the user selects a route that does not meet the user's purpose, we give a penalty value.
By repeating this process, the reinforcement learning algorithm may be able to select a route that achieves a better goal.

Through these experiments, we hope to show that an approach using reinforcement learning, which is a kind of algorithm that makes independent decisions without correct data, is effective for making decisions that satisfy human objectives and is practical for path selection of mobility.
~ \\
Keywords : \\
\underline{1. Machine Learning},
\underline{2. Reinforcement learning},
\underline{3. Mobility},
\underline{4. Sharing Economy}
\begin{flushright}
\edept \\
\eauthor
\end{flushright}
