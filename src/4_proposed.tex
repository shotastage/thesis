\chapter{提案手法}
\label{proposed}

本章では提案手法について述べる.

\section{概要}

本研究では, 人間の状態とモビリティの経路を再現するシミュレータープログラムを開発し実験を行う.
なお, シミュレータープログラムの作成にあたっては以下について仮定を行った.

\begin{itemize}
    \item シミュレーター上での自動車は全てレベル5の完全自動運転ができる物とする
    \item 歩行者や, 自動運転非対応者などは考慮していない
    \item シミュレーターでは予め決められた幹線道路及, 環状線などのバイパス道路のみを考慮し, それ以外の道に関しては考慮しない
    \item 人間の目的を正確に推定するコンピューターシステムもしくはセンサーなどは実在しているものとする. 
\end{itemize}

\subsection{路線網の正規化}

本研究では実際の道路網を参考に仮想の道路網モデルを構築した.


\begin{figure}[H]
    \centering  % 図を真ん中に配置
    \includegraphics[clip,width = 13.0cm]{assets/map_origin.eps}
    \caption{ベースとなった場所付近の地図}  \label{sample}
\end{figure}


\subsection{選択した道路のデータ化}

選択した道路の線形をPostgreSQLにてデータ化を行った. なお, この実験で使用したPostgreSQLには地理情報を扱う拡張であるPostGISをインストールしている.
データ化した内容は, 道路網の一定間隔の座標である. この座標同士を結ぶ線分をPOLYLINE型で記憶し路線名や道路が受け入れることのできる車の第数=Capacityを定義した.

\begin{lstlisting}[caption = 路線データを表すクエリーの例, label = program1]

/*
    テーブル構造の定義
*/
CREATE TABLE geodb_catp_yokohama (
    id                  serial PRIMARY KEY,
    route_name          VARCHAR (100),
    has_spot    VARCHAR (20),
    capacity            INTEGER
);

/*
    線地理情報を記録するPOLYLINE型のカラムを追加
*/
SELECT AddGeometryColumn ('public', 'geodb_catp_yokohama', 'geometry_data', 4326, 'LINESTRING', 2);

/*
    道路の線形を表すデータのレコード
*/
INSERT INTO geodb_catp_yokohama (route_name, has_spot, capacity, geometry_data)
VALUES ('K1',
        'none',
        200,
        ST_GeomFromText('LINESTRING(139.633639 35.445911, 139.632631 35.447126 ...... 139.632323 35.472085)', 4326));
\end{lstlisting}

    
    
    


%%% Local Variables:
%%% mode: japanese-latex
%%% TeX-master: "../bthesis"
%%% End:
